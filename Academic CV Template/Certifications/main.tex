\section{\textbf{Certifications}}
\vspace{-0.4mm}
\resumeSubHeadingListStart

\resumePOR{Google: }{\href{https://coursera.org/share/325c2d8b438498676ea7544f96a6fff8}{
\textbf{Google Cybersecurity Professional Certificate}
}}{June 2023}

\resumePOR{Standford University: }{\href{https://www.coursera.org/account/accomplishments/specialization/WEHNE78JGJ44}{
\textbf{Machine Learning Specialization}
}}{July 2024}
% \resumePOR{}{
% \textbf{Google: } {{\href{https://www.coursera.org/account/accomplishments/professional-cert/3B59P6PJ593T?utm_source=link&utm_medium=certificate&utm_content=cert_image&utm_campaign=sharing_cta&utm_product=prof}{Google Cybersecurity Professional Certificate}}}}{Month Year}
% \resumePOR{}{
% \textbf{Certifying Body:} {{\href{https://certification-link-c.com}{Certification C}}}}{Month Year}

\resumePOR{Udemy: }{\href{https://www.udemy.com/course/learn-flutter-dart-to-build-ios-android-apps/}{
\textbf{Flutter \& Dart - The Complete Guide}
}}{In progress}

\resumePOR{Udemy: }{\href{https://www.udemy.com/course/master-linux-administration/}{
\textbf{Linux Administration: The Complete Linux Bootcamp}
}}{In progress}

\resumePOR{Udemy: }{\href{https://www.udemy.com/course/ccna-complete}{
\textbf{Cisco CCNA 200-301 - The Complete Guide}
}}{In progress}

\resumeSubHeadingListEnd
\vspace{-6mm}